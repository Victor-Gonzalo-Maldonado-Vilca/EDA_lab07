%package list
\documentclass{article}
\usepackage[top=3cm, bottom=3cm, outer=3cm, inner=3cm]{geometry}
\usepackage{multicol}
\usepackage{graphicx}
\usepackage{url}
%\usepackage{cite}
\usepackage{hyperref}
\usepackage{array}
%\usepackage{multicol}
\newcolumntype{x}[1]{>{\centering\arraybackslash\hspace{0pt}}p{#1}}
\usepackage{natbib}
\usepackage{pdfpages}
\usepackage{multirow}    
\usepackage[normalem]{ulem}
\useunder{\uline}{\ul}{}
\usepackage{svg}
\usepackage{xcolor}
\usepackage{listings}
\lstdefinestyle{ascii-tree}{
    literate={├}{|}1 {─}{--}1 {└}{+}1 
  }

\lstset{basicstyle=\ttfamily,
  showstringspaces=false,
  commentstyle=\color{red},
  keywordstyle=\color{blue}
}
%\usepackage{booktabs}
\usepackage{caption}
\usepackage{subcaption}
\usepackage{float}
\usepackage{array}

\usepackage{enumitem}


\newcolumntype{M}[1]{>{\centering\arraybackslash}m{#1}}
\newcolumntype{N}{@{}m{0pt}@{}}


%%%%%%%%%%%%%%%%%%%%%%%%%%%%%%%%%%%%%%%%%%%%%%%%%%%%%%%%%%%%%%%%%%%%%%%%%%%%
%%%%%%%%%%%%%%%%%%%%%%%%%%%%%%%%%%%%%%%%%%%%%%%%%%%%%%%%%%%%%%%%%%%%%%%%%%%%
\newcommand{\itemEmail}{vmaldonadov@unsa.edu.pe, acunoc@unsa.edu.pe}
\newcommand{\itemStudent}{Victor Gonzalo Maldonado Vilca, Armando Steven Cuno Cahuari}
\newcommand{\itemCourse}{Estructura de Datos y Algoritmos}
\newcommand{\itemCourseCode}{1702122}
\newcommand{\itemSemester}{III}
\newcommand{\itemUniversity}{Universidad Nacional de San Agustín de Arequipa}
\newcommand{\itemFaculty}{Facultad de Ingeniería de Producción y Servicios}
\newcommand{\itemDepartment}{Departamento Académico de Ingeniería de Sistemas e Informática}
\newcommand{\itemSchool}{Escuela Profesional de Ingeniería de Sistemas}
\newcommand{\itemAcademic}{2024 - A}
\newcommand{\itemInput}{Del 04/07/24 -- 09:42am}
\newcommand{\itemOutput}{Al 04/07/24 -- 23:59pm}
\newcommand{\itemPracticeNumber}{07}
\newcommand{\itemTheme}{Trie}
%%%%%%%%%%%%%%%%%%%%%%%%%%%%%%%%%%%%%%%%%%%%%%%%%%%%%%%%%%%%%%%%%%%%%%%%%%%%
%%%%%%%%%%%%%%%%%%%%%%%%%%%%%%%%%%%%%%%%%%%%%%%%%%%%%%%%%%%%%%%%%%%%%%%%%%%%

\usepackage[english,spanish]{babel}
\usepackage[utf8]{inputenc}
\AtBeginDocument{\selectlanguage{spanish}}
\renewcommand{\figurename}{Figura}
\renewcommand{\refname}{Referencias}
\renewcommand{\tablename}{Tabla} %esto no funciona cuando se usa babel
\AtBeginDocument{%
	\renewcommand\tablename{Tabla}
}

\usepackage{fancyhdr}
\pagestyle{fancy}
\fancyhf{}
\setlength{\headheight}{30pt}
\renewcommand{\headrulewidth}{1pt}
\renewcommand{\footrulewidth}{1pt}
\fancyhead[L]{\raisebox{-0.2\height}{\includegraphics[width=3cm]{img/logo_episunsa.png}}}
\fancyhead[C]{\fontsize{7}{7}\selectfont	\itemUniversity \\ \itemFaculty \\ \itemDepartment \\ \itemSchool \\ \textbf{\itemCourse}}
\fancyhead[R]{\raisebox{-0.2\height}{\includegraphics[width=1.2cm]{img/logo_abet}}}
\fancyfoot[L]{Victor M.}
\fancyfoot[C]{\itemCourse}
\fancyfoot[R]{Página \thepage}

% para el codigo fuente
\usepackage{listings}
\usepackage{color, colortbl}
\definecolor{dkgreen}{rgb}{0,0.6,0}
\definecolor{gray}{rgb}{0.5,0.5,0.5}
\definecolor{mauve}{rgb}{0.58,0,0.82}
\definecolor{codebackground}{rgb}{0.95, 0.95, 0.92}
\definecolor{tablebackground}{rgb}{0.8, 0, 0}

\lstset{frame=tb,
	language=bash,
	aboveskip=3mm,
	belowskip=3mm,
	showstringspaces=false,
	columns=flexible,
	basicstyle={\small\ttfamily},
	numbers=none,
	numberstyle=\tiny\color{gray},
	keywordstyle=\color{blue},
	commentstyle=\color{dkgreen},
	stringstyle=\color{mauve},
	breaklines=true,
	breakatwhitespace=true,
	tabsize=3,
	backgroundcolor= \color{codebackground},
}

\begin{document}
	
	\vspace*{10px}
	
	\begin{center}	
		\fontsize{17}{17} \textbf{ Informe de Laboratorio 07}
	\end{center}
	\centerline{\textbf{\Large Tema: \itemTheme}}
	%\vspace*{0.5cm}	

	\begin{flushright}
		\begin{tabular}{|M{2.5cm}|N|}
			\hline 
			\rowcolor{tablebackground}
			\color{white} \textbf{Nota}  \\
			\hline 
			     \\[30pt]
			\hline 			
		\end{tabular}
	\end{flushright}	

	\begin{table}[H]
		\begin{tabular}{|x{4.7cm}|x{4.8cm}|x{4.8cm}|}
			\hline 
			\rowcolor{tablebackground}
			\color{white} \textbf{Estudiante} & \color{white}\textbf{Escuela}  & \color{white}\textbf{Asignatura}   \\
			\hline 
			{\itemStudent \par \itemEmail} & \itemSchool & {\itemCourse \par Semestre: \itemSemester \par Código: \itemCourseCode}     \\
			\hline 			
		\end{tabular}
	\end{table}		
	
	\begin{table}[H]
		\begin{tabular}{|x{4.7cm}|x{4.8cm}|x{4.8cm}|}
			\hline 
			\rowcolor{tablebackground}
			\color{white}\textbf{Tarea} & \color{white}\textbf{Tema}  & \color{white}\textbf{Duración}   \\
			\hline 
			\itemPracticeNumber & \itemTheme & 2 horas   \\
			\hline 
		\end{tabular}
	\end{table}
	
	\begin{table}[H]
		\begin{tabular}{|x{4.7cm}|x{4.8cm}|x{4.8cm}|}
			\hline 
			\rowcolor{tablebackground}
			\color{white}\textbf{Semestre académico} & \color{white}\textbf{Fecha de inicio}  & \color{white}\textbf{Fecha de entrega}   \\
			\hline 
			\itemAcademic & \itemInput &  \itemOutput  \\
			\hline 
		\end{tabular}
	\end{table}
%%%%%%%%%%%%%%%%%%%%

  \section{Introducción}
  Los árboles Trie son estructuras de datos eficientes para almacenar y recuperar cadenas de caracteres, 
  especialmente útiles para operaciones rápidas de búsqueda y autocompletado. En este documento, se explora la 
  implementación de un árbol Trie en Java, enfocándose en su diseño y operaciones fundamentales.

%%%%%%%%%%%%%%%%%%%%

  \section{Objetivos}
  \begin{itemize}
  \item \textbf{Implementación del Árbol Trie:} Desarrollar un árbol Trie eficiente en Java, incluyendo métodos para inserción, 
  búsqueda y eliminación de cadenas.
  \item \textbf{Análisis de Complejidad:} Evaluar la complejidad temporal y espacial de las operaciones del árbol Trie implementado.
  \item \textbf{Aplicaciones Prácticas:} Explorar aplicaciones como autocompletado de texto y gestión de listas de palabras clave usando 
  árboles Trie.
  \end{itemize}

%%%%%%%%%%%%%%%%%%%%
 
	\section{Tarea}
  \begin{itemize}
    \item Elabore un informe paso a paso de la implementación un Trie para insertar, buscar y reemplazar palabras en un texto.
    \item Encuentra las primeras 'k' palabras que ocurren con mayor frecuencia en un conjunto dado de cadenas(que se insertaron previamente en el Trie)
  \end{itemize}
 
%%%%%%%%%%%%%%%%%%%% 
 
  \section{Entregables}
  \begin{itemize}
    \item Informe hecho en Latex.
    \item URL: Repositorio GitHub.
    \item Archivos Java. 
  \end{itemize}
  
%%%%%%%%%%%%%%%%%%%%    
		
	\section{Equipos, materiales y temas utilizados}
  \begin{itemize}
    \item Trie
    \item Git
    \item notepad++
    \item Latex
    \item Java
  \end{itemize}
  
%%%%%%%%%%%%%%%%%%%%

  \section{URL de Repositorio Github}
  \begin{itemize}
    \item Link: GitHub.
    \item \url{https://github.com/Victor-Gonzalo-Maldonado-Vilca/EDA_lab07.git}
  \end{itemize}

%%%%%%%%%%%%%%%%%%%%

  \section{Desarrollo del trabajo}
  
%%%%%%%%%%%%

  \subsection{TrieNode}
  \begin{itemize}
    \item \textbf{Descripción: }La clase TrieNode representa un nodo en un trie, una estructura de datos utilizada para almacenar 
    cadenas de caracteres. Cada nodo tiene dos atributos: children, que es un mapa que asocia cada carácter a su nodo hijo 
    correspondiente, y frequency, un entero que cuenta cuántas veces una palabra termina en este nodo. El constructor inicializa 
    el mapa de hijos como un HashMap vacío y la frecuencia a 0. Esta clase es fundamental para construir y operar sobre un trie.
    \item \textbf{Código: }
    \begin{lstlisting}[language=Java, caption={Clase TrieNode}]
      import java.util.*;
      class TrieNode {
          Map<Character, TrieNode> children;
          int frequency;

          public TrieNode() {
              children = new HashMap<>();
              frequency = 0;
          }
      }
    \end{lstlisting}
  \end{itemize}
  
%%%%%%%%%%%%

  \subsection{Trie}
  
%%%%%%
  
  \subsubsection{Definición de clase Trie}
  \begin{itemize}
    \item \textbf{Descripción: }Esta sección define la clase Trie y su constructor. La clase contiene un nodo raíz de tipo TrieNode.
    \item \textbf{Código: }
    \begin{lstlisting}[language=Java, caption={Clase Trie}]
        import java.util.*;
        class Trie {
            private TrieNode root;

            public Trie() {
                root = new TrieNode();
            }
    \end{lstlisting}
  \end{itemize}
    
%%%%%%
  
  \subsubsection{Método Insertar}
  \begin{itemize}
    \item \textbf{Descripción: }Este método inserta una palabra en el trie. Recorre cada carácter de la palabra y actualiza los nodos correspondientes. Si un carácter no existe, 
    se crea un nuevo nodo.
    \item \textbf{Código: } 
    \begin{lstlisting}[language=Java, caption={Método Insertar}]
        public void insert(String word) {
            TrieNode node = root;
            for (char c : word.toCharArray()) {
                if (!node.children.containsKey(c)) {
                    node.children.put(c, new TrieNode());
                }
                node = node.children.get(c);
            }
            node.frequency++;
        }
    \end{lstlisting}
  \end{itemize}
    
%%%%%%
  
  \subsubsection{Método Search}
  \begin{itemize}
    \item \textbf{Descripción: }Este método busca una palabra en el trie. Devuelve true si la palabra existe (es decir, si su frecuencia es mayor que 0), 
    de lo contrario, devuelve false.
    \item \textbf{Código: }
    \begin{lstlisting}[language=Java, caption={Método Search}]
        public boolean search(String word) {
            TrieNode node = root;
            for (char c : word.toCharArray()) {
                if (!node.children.containsKey(c)) {
                    return false;
                }
                node = node.children.get(c);
            }
            return node.frequency > 0;
        }
    \end{lstlisting}
  \end{itemize}
    
    
%%%%%%
  
  \subsubsection{Método replaceWords}
  \begin{itemize}
    \item \textbf{Descripción: }Este método reemplaza todas las palabras en un texto que se encuentran en el trie con una palabra de reemplazo dada. 
    Devuelve el texto modificado.
    \item \textbf{Código: }
    \begin{lstlisting}[language=Java, caption={Método replaceWords}]
        public String replaceWords(String text, String replacement) {
            String[] words = text.split("\\s+");
            StringBuilder result = new StringBuilder();

            for (String word : words) {
                if (search(word)) {
                    result.append(replacement).append(" ");
                } else {
                    result.append(word).append(" ");
                }
            }

            return result.toString().trim();
        }
    \end{lstlisting}
  \end{itemize}
    
%%%%%%
  
  \subsubsection{Método getTopKFrequentWords}
  \begin{itemize}
    \item \textbf{Descripción: }Este método devuelve una lista con las k palabras más frecuentes en el trie. Utiliza un mapa para almacenar las frecuencias y 
    las ordena para encontrar las palabras más comunes.
    \item \textbf{Código: }
    \begin{lstlisting}[language=Java, caption={Método getTopKFrequentWords}]
        public List<String> getTopKFrequentWords(int k) {
            // Usamos un mapa para almacenar las frecuencias de las palabras
            Map<String, Integer> wordFrequencyMap = new HashMap<>();
            collectWords(root, "", wordFrequencyMap);

            // Convertimos el mapa a una lista de entradas y ordenamos por frecuencia
            List<Map.Entry<String, Integer>> entries = new ArrayList<>(wordFrequencyMap.entrySet());
            entries.sort((a, b) -> b.getValue() - a.getValue());

            // Obtenemos las primeras k palabras mas frecuentes
            List<String> topKWords = new ArrayList<>();
            for (int i = 0; i < k && i < entries.size(); i++) {
                topKWords.add(entries.get(i).getKey());
            }

            return topKWords;
        }
    \end{lstlisting}
  \end{itemize}
  
%%%%%%
  
  \subsubsection{Método collectWords}
  \begin{itemize}
    \item \textbf{Descripción: } Este método auxiliar recolecta todas las palabras en el trie junto con sus frecuencias. Se usa recursivamente para recorrer todos los nodos del trie y 
    actualizar el mapa de frecuencias.
    \item \textbf{Código: }
    \begin{lstlisting}[language=Java, caption={Método collectWords}]
        private void collectWords(TrieNode node, String prefix, Map<String, Integer> wordFrequencyMap) {
            if (node == null) {
                return;
            }
            if (node.frequency > 0) {
                wordFrequencyMap.put(prefix, node.frequency);
            }
            for (Map.Entry<Character, TrieNode> entry : node.children.entrySet()) {
                collectWords(entry.getValue(), prefix + entry.getKey(), wordFrequencyMap);
            }
            
        }
    \end{lstlisting}
  \end{itemize}
  
%%%%%%%%%%%%

  \subsection{TopKFrequendtWords}
  \begin{itemize}
    \item \textbf{Descripción: }El código define una clase TopKFrequentWords que utiliza la estructura de datos Trie para realizar varias operaciones relacionadas con 
    palabras y sus frecuencias. La clase Trie se utiliza para insertar, buscar y reemplazar palabras, así como para obtener las palabras 
    más frecuentes. El método main de la clase TopKFrequentWords se encarga de demostrar estas funcionalidades.
    \item \textbf{Código: }
    \begin{lstlisting}[language=Java, caption={Clase Principal}]
        import java.util.*;
        public class TopKFrequentWords {
             public static void main(String[] args) {
                Trie trie = new Trie();
                trie.insert("will");
                trie.insert("win");
                trie.insert("wish");
                trie.insert("war");
                trie.insert("war");
                trie.insert("war");
                trie.insert("war");
                trie.insert("want");
                trie.insert("warp");
                trie.insert("warp");
                trie.insert("warp");
                trie.insert("want");
                trie.insert("warp");
                trie.insert("wee");
                trie.insert("wee");
                
                System.out.println("Search 'will': " + trie.search("will")); 
                System.out.println("Search 'wee': " + trie.search("wee")); 
                System.out.println("Search 'hi': " + trie.search("hi")); 
                System.out.println("Search 'wanted': " + trie.search("wanted")); 

                String text = "hello";
                String replacedText = trie.replaceWords(text, "warp");
                System.out.println("Replaced Text: " + replacedText); // REPLACED REPLACED REPLACED hola

                List<String> topWords = trie.getTopKFrequentWords(2);
                System.out.println("Top 2 Frequent Words: " + topWords); // [hello, world]
            }
        }
    \end{lstlisting}
    \item \textbf{Ejecución: }
    \begin{figure}[H]
      \centering
      \includegraphics[width=1\textwidth, keepaspectratio]{img/cmd1.png}
      \caption{Trie -- Ejecución}
    \end{figure}
  \end{itemize}

%%%%%%%%%%%%%%%%%%%%

  \section{Conclusiones}
  \begin{enumerate}
    \item \textbf{Eficiencia}: Los tries son eficientes para buscar y almacenar cadenas, con operaciones de inserción, búsqueda 
    y eliminación en tiempo proporcional a la longitud de la cadena.
    \item \textbf{Autocompletado}: Son ideales para autocompletado y sugerencias de palabras, permitiendo búsquedas rápidas de prefijos.  
    \item \textbf{Memoria}: Pueden consumir más memoria debido a múltiples nodos y punteros, pero evitan la duplicación de 
    prefijos comunes.  
    \item \textbf{Frecuencias}: Pueden manejar frecuencias de palabras para análisis estadísticos y encontrar palabras comunes.
    \item \textbf{Aplicaciones}: Se utilizan en corrección ortográfica, procesamiento de lenguaje natural y compresión de datos.   
    \item \textbf{Flexibilidad}: Su implementación es simple y se puede extender para incluir información adicional, como 
    frecuencias de palabras.
  \end{enumerate}

%%%%%%%%%%%%%%%%%%%%
	\newpage
	\subsection{\textcolor{red}{Rúbrica para el contenido del Informe y demostración}}
	\begin{itemize}			
		\item El alumno debe marcar o dejar en blanco en celdas de la columna \textbf{Checklist} si cumplio con el ítem correspondiente.
		\item Si un alumno supera la fecha de entrega,  su calificación será sobre la nota mínima aprobada, siempre y cuando cumpla con todos lo items.
		\item El alumno debe autocalificarse en la columna \textbf{Estudiante} de acuerdo a la siguiente tabla:
	
		\begin{table}[ht]
			\caption{Niveles de desempeño}
			\begin{center}
			\begin{tabular}{ccccc}
    			\hline
    			 & \multicolumn{4}{c}{Nivel}\\
    			\cline{1-5}
    			\textbf{Puntos} & Insatisfactorio 25\%& En Proceso 50\% & Satisfactorio 75\% & Sobresaliente 100\%\\
    			\textbf{2.0}&0.5&1.0&1.5&2.0\\
    			\textbf{4.0}&1.0&2.0&3.0&4.0\\
    		\hline
			\end{tabular}
		\end{center}
	\end{table}	
	

	\end{itemize}

 
	
	\begin{table}[H]
		\caption{Rúbrica para contenido del Informe y demostración}
		\setlength{\tabcolsep}{0.5em} % for the horizontal padding
		{\renewcommand{\arraystretch}{1.5}% for the vertical padding
		%\begin{center}
		\begin{tabular}{|p{2.7cm}|p{7cm}|x{1.3cm}|p{1.2cm}|p{1.5cm}|p{1.1cm}|}
			\hline
    		\multicolumn{2}{|c|}{Contenido y demostración} & Puntos & Checklist & Estudiante & Profesor\\
			\hline
			\textbf{1. GitHub} & Hay enlace URL activo del directorio para el  laboratorio hacia su repositorio GitHub con código fuente terminado y fácil de revisar. &2 &X &2 & \\ 
			\hline
			\textbf{2. Commits} &  Hay capturas de pantalla de los commits más importantes con sus explicaciones detalladas. (El profesor puede preguntar para refrendar calificación). &4 &X &4 & \\ 
			\hline 
			\textbf{3. Código fuente} &  Hay porciones de código fuente importantes con numeración y explicaciones detalladas de sus funciones. &2 &X &2 & \\ 
			\hline 
			\textbf{4. Ejecución} & Se incluyen ejecuciones/pruebas del código fuente  explicadas gradualmente. &2 &X &2 & \\ 
			\hline			
			\textbf{5. Pregunta} & Se responde con completitud a la pregunta formulada en la tarea.  (El profesor puede preguntar para refrendar calificación).  &2 &X &2 & \\ 
			\hline	
			\textbf{6. Fechas} & Las fechas de modificación del código fuente estan dentro de los plazos de fecha de entrega establecidos. &2 &X &2 & \\ 
			\hline 
			\textbf{7. Ortografía} & El documento no muestra errores ortográficos. &2 &X &2 & \\ 
			\hline 
			\textbf{8. Madurez} & El Informe muestra de manera general una evolución de la madurez del código fuente,  explicaciones puntuales pero precisas y un acabado impecable.   (El profesor puede preguntar para refrendar calificación).  &4 &X &4 & \\ 
			\hline
			\multicolumn{2}{|c|}{\textbf{Total}} &20 & &20 & \\ 
			\hline
		\end{tabular}
		%\end{center}
		%\label{tab:multicol}
		}
	\end{table}


%%%%%%%%%%%%%%%%%%%%%%%%%%%%%%%%%%%%%%%%%%%%%%%%%%%%%%%%%%%%%%%%%%%
	
  \newpage
  \section{Referencias}
  \begin{itemize}
    \item \url{https://www.baeldung.com/trie-java}
    \item \url{https://www.youtube.com/watch?v=fUpZ05dNZdE}
    \item \url{https://www.w3schools.com/java/}
    \item \url{https://www.eclipse.org/downloads/packages/release/2022-03/r/eclipse-ide-enterprise-java-and-web-developers}
    \item \url{https://docs.oracle.com/javase/7/docs/api/java/util/List.html}
    \item \url{https://docs.oracle.com/javase/tutorial/java/generics/types.html}
  \end{itemize}

%%%%%%%%%%%%%%%%%%%% 
%\clearpage
%\bibliographystyle{apalike}
%\bibliographystyle{IEEEtranN}
%\bibliography{bibliography}
			
\end{document}
